\documentclass{article}
 
\usepackage[margin=4cm]{geometry}
\usepackage{physics, amssymb}
\usepackage[utf8]{inputenc}
\usepackage[english]{babel}
\usepackage{subcaption}
\usepackage{graphicx}
\usepackage{parskip}
\usepackage{minted}
\usepackage{esint}
\usepackage{hyperref} 
\hypersetup{
	colorlinks=true,
	linkcolor=blue,
	filecolor=magenta,
	urlcolor=cyan,
	pdftitle={Overleaf Example},
	pdfpagemode=FullScreen,
}

\begin{document}

\author{Emilie Dørum}
\title{FYS-STK4155 - Project 1}

\date{\today}
\maketitle

\newcommand{\dunder}[1] {
	\underline{\underline{#1}} 
}

\newcommand{\mat}[1] {
	\begin{bmatrix}#1\end{bmatrix}
}

\newcommand{\bs}[1] {
	\boldsymbol{#1}
}

\newcommand{\bhat}[1] {
	\hat{\boldsymbol{#1}}
}

\newcommand{\vary}[1] {
	\text{Var}\left(#1\right)
}

\newcommand{\expy}[1] {
	\mathbb{E}\left(#1\right)
}

\newcommand{\avg}[1] {
	\left<#1\right>
}

\newcommand{\nroot}[2][2]{
	\sqrt[\uproot{2}\scriptstyle #1]{#2}
}

\newcommand{\navsection}[1]{
	\section*{#1}
	\addcontentsline{toc}{section}{#1}
}
\newcommand{\navsubsection}[1]{
	\subsection*{#1}
	\addcontentsline{toc}{subsection}{#1} 
}
\newcommand{\navsubsubsection}[1]{
	\subsubsection*{#1}
	\addcontentsline{toc}{subsubsection}{#1}
}

\definecolor{bg}{rgb}{0.95, 0.95, 0.95}

\navsection{Temp}
Franke's function:
\begin{align*}
    f(x, y) &= \frac{3}{4}\exp(-\frac{(9x-2)^2}{4} - \frac{(9y-2)^2}{4}) + \frac{3}{4}\exp(-\frac{(9x+1)^2}{49} - \frac{9y+1}{10}) \\
    &+ \frac{1}{2}\exp(-\frac{(9x-7)^2}{4} - \frac{(9y-3)^2}{4}) - \frac{1}{5}\exp(-(9x-4)^2 - (9y-7)^2)
\end{align*}
$x \in [0, 1]$

\navsubsection{Part a)}
We assume that there exists a continuous function $f(\bs{x})$
and a normal distributed error $\bs{\varepsilon} \sim N(0, \sigma^2)$ such that our data is described by
$$\bs{y} = f(\bs{x}) + \bs{\varepsilon}$$
We want to approximate this function with our model from the solution of the
ordinary least squares (OLS) linear regression equations. This means that
our function $f$ is approximated by $\bs{\tilde{y}}$ where we minimized 
$(\bs{y} - \bs{\tilde{y}})^2$, with
$$\bs{\tilde{y}} = \bs{X}\bs{\beta}$$
Where $\bs{X}$ is the design matrix.

We want to show that the expectation value of $\bs{y}$ for a given element $i$ is
$$\expy{y_i} = \sum_j x_{ij}\beta_j = (\bs{X}\bs{\beta})_i$$
and that its variance is
$$\vary{y_i} = \sigma^2$$
We use the definition of $y_i$, $y_i = (\bs{X}\bs{\beta})_i + \epsilon_i$. This gives
\begin{align*}
	\expy{y_i} &= \expy{(\bs{X}\bs{\beta})_i + \epsilon_i} \\
	&= \expy{(\bs{X}\bs{\beta})_i} + \expy{\epsilon_i} \\
\end{align*}
We assume that our x-values are not stocastic. Therefore $\expy{(\bs{X}\bs{\beta})_i} = (\bs{X}\bs{\beta})_i$.
We also assume that $\bs{\varepsilon} \sim N(0, \sigma^2)$, so $\expy{\epsilon_i} = 0$.
We therefore get $\expy{y_i} = (\bs{X}\bs{\beta})_i\ \ \ \square$. \\
To find $\vary{y_i}$ we use the definition of the variance, $\vary{x} = \expy{(x - \expy{x})^2}$. This gives
\begin{align*}
	\vary{y_i} &= \expy{(y_i - \expy{y_i})^2} \\
	&= \expy{((\bs{X}\bs{\beta})_i + \epsilon_i - (\bs{X}\bs{\beta})_i)^2} \\
	&= \expy{\epsilon_i^2} \\
	&= \sigma^2\ \ \ \square \\
\end{align*}
Where we used our previous result and our assumption that $\bs{\varepsilon} \sim N(0, \sigma^2)$.


We now want to find the expectation value and variance of $\bs{\hat{\beta}}$.
We know that the $\bs{\hat{\beta}}$ that minimizes
$(\bs{y} - \bs{\tilde{y}})^2$ is 
$$\bs{\hat{\beta}} = \qty(\bs{X}^T\bs{X})^{-1}\bs{X}^T\bs{y}$$
Using this we can find the expectation value and variance:
\begin{align*}
	\expy{\bs{\hat{\beta}}} &= \mathbb{E}\qty(\qty(\bs{X}^T\bs{X})^{-1}\bs{X}^T\bs{y}) \\
	&= \qty(\bs{X}^T\bs{X})^{-1}\bs{X}^T\expy{\bs{y}} \\
	&= \qty(\bs{X}^T\bs{X})^{-1}\bs{X}^T\bs{X}\bs{\beta} \\
	&= \bs{\beta} \\
	\vary{\bs{\hat{\beta}}} &= \mathbb{E}\qty((\bs{\hat{\beta}} - \expy{\bs{\hat{\beta}}})^2) \\
	&= \mathbb{E}\qty((\bs{\hat{\beta}} - \bs{\beta})^2) \\
	&= \mathbb{E}\qty(\qty(\qty(\bs{X}^T\bs{X})^{-1}\bs{X}^T\bs{y} - \bs{\beta})^2) \\
	&= \mathbb{E}\qty(\qty(\qty(\bs{X}^T\bs{X})^{-1}\bs{X}^T(\bs{X}\bs{\beta} + \bs{\epsilon}) - \bs{\beta})^2) \\
	&= \mathbb{E}\qty(\qty(\bs{\beta} + \qty(\bs{X}^T\bs{X})^{-1}\bs{X}^T\bs{\epsilon} - \bs{\beta})^2) \\
	&= \qty(\qty(\bs{X}^T\bs{X})^{-1}\bs{X}^T)^2\mathbb{E}\qty(\bs{\epsilon}^2) \\
	&= \qty(\qty(\bs{X}^T\bs{X})^{-1}\bs{X}^T)^2\sigma^2 \\
	&= \sigma^2\qty(\bs{X}^T\bs{X})^{-1} \\
\end{align*}

\navsection{Abstract}
The abstract gives the reader a quick overview of what has been done and the most important results. Try to be to the point and state your main findings. It could be structured as follows o Short introduction to topic and why its important o Introduce a challenge or unresolved issue with the topic (that you will try to solve) o What have you done to solve this o Main Results o The implications
The report, the introduction

\navsection{Introduction}
Motivate the reader, the first part of the introduction gives always a motivation and tries to give the overarching ideas
What I have done
The structure of the report, how it is organized etc

\navsection{Method}
Describe the methods and algorithms
You need to explain how you implemented the methods and also say something about the structure of your algorithm and present some parts of your code
You should plug in some calculations to demonstrate your code, such as selected runs used to validate and verify your results. The latter is extremely important!! A reader needs to understand that your code reproduces selected benchmarks and reproduces previous results, either numerical and/or well-known closed form expressions.

\navsection{Results}
Present your results
Give a critical discussion of your work and place it in the correct context.
Relate your work to other calculations/studies
An eventual reader should be able to reproduce your calculations if she/he wants to do so. All input variables should be properly explained.
Make sure that figures and tables should contain enough information in their captions, axis labels etc so that an eventual reader can gain a first impression of your work by studying figures and tables only.

\navsection{Conclusion}
    State your main findings and interpretations
    Try as far as possible to present perspectives for future work
    Try to discuss the pros and cons of the methods and possible improvements

\navsection{Appendix}
Additional calculations used to validate the codes
Selected calculations, these can be listed with few comments
Listing of the code if you feel this is necessary

You can consider moving parts of the material from the methods section to the appendix. You can also place additional material on your webpage or GitHub page..

\navsection{References}
Give always references to material you base your work on, either scientific articles/reports or books.
Refer to articles as: name(s) of author(s), journal, volume (boldfaced), page and year in parenthesis.
Refer to books as: name(s) of author(s), title of book, publisher, place and year, eventual page numbers


\end{document}
